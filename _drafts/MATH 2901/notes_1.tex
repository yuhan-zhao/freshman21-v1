\documentclass[11pt]{article}
\usepackage[utf8]{inputenc}
\usepackage[shortlabels]{enumitem}
\usepackage{amsmath}
\usepackage{amssymb}
\usepackage{amsthm}
\usepackage{xcolor}
\usepackage{graphicx}
\usepackage{blkarray}

\usepackage{hyperref}
\hypersetup{
    colorlinks=true,
    linkcolor=blue,
    filecolor=magenta,      
    urlcolor=cyan,
}


% setting page format
\topmargin -.5in
\textheight 9in
\oddsidemargin -.25in
\evensidemargin -.25in
\textwidth 7in
\setlength{\parindent}{0 in}
\setlength{\parskip}{0.1 in}

% setting new environments
\newtheorem*{theorem}{Theorem}
\newtheorem*{lemma}{Lemma}
\newtheorem*{corollary}{Corollary}

\theoremstyle{definition}
\newtheorem*{definition}{Definition}
\newtheorem*{example}{Example}
\newtheorem*{problem}{Problem}
\newtheorem*{question}{Question}

\theoremstyle{remark}
\newtheorem*{remark}{Remark}

\newenvironment{solution}[1][Solution]{\textbf{#1:} \par}{\ $\blacksquare$}

\newenvironment{newnotion}[1]{\textbf{#1.}}

\newenvironment{caution}{$\bigstar\bigstar\bigstar$\textbf{Caution.}}

%\newenvironment{proof}[1][Proof]{\textbf{#1:} \par}{\ \rule{0.5em}{0.5em}}
%\newenvironment{problem}[1]{\textbf{#1:} }

% define new commands
\renewcommand{\hat}{\widehat}
\renewcommand{\tilde}{\widetilde}

\newcommand{\numpy}{{\tt numpy}}    % tt font for numpy
\newcommand{\dom}[1]{\mathbf dom #1}
\newcommand{\tr}{\mathbf{tr}}
\newcommand{\sgn}{\text{sgn}}
\newcommand{\spacevert}{\;\vert\;}
\newcommand{\ie}{i.e.}
\newcommand{\eg}{e.g.}

\newcommand{\N}{\mathbb{N}}
\newcommand{\E}{\mathbb{{E}}}
\newcommand{\Q}{\mathbb{Q}}
\newcommand{\R}{\mathbb{R}}
\newcommand{\C}{\mathbb{C}}
\newcommand{\Z}{\mathbb{Z}}
\newcommand{\mS}{\mathbb{S}}

% for probability
\newcommand{\Exp}{\mathbf{E}}
\newcommand{\Var}{\mathbf{Var}}
\newcommand{\Prob}{\mathbf{P}}
\newcommand{\card}{\mathbf{card}}
\newcommand{\cov}{\mathbf{cov}}
\newcommand{\corr}{\mathbf{corr}}
\newcommand{\1}{\mathbf{1}}


% for optimizations
\newcommand{\subto}{\text{s.t.}}

% for norms
\newcommand{\norm}[1]{\left\Vert #1 \right\Vert}
\newcommand{\abs}[1]{\left\vert #1 \right\vert}

\begin{document}
% ========== Edit your name here
\title{MATH 2901 Basic Probability Lecture Notes 1}
\author{Instructor: Richard Kleeman}
\date{}
\maketitle

%\medskip

% ========== Contents begin here ==============
\section{Event as sets}
Probability studies the \emph{repeatable (or ideal) experiments}. The result of an experiment is called an \emph{outcome}.

\begin{definition}
The set of all possible outcomes of an experiment is called the \textbf{sample space} and is denoted by $\Omega$.  
\end{definition}

\subsection{Cardinality of sets}
The cardinality of a set $\Omega$ refers to the number of the number of elements in this set, and is denoted by $\card(\Omega)$ or $\abs{\Omega}$.
\begin{itemize}
    \item For finite sets, $\card(\Omega)$ is a natural number.
    \item For the integer set, $\card(\Z) = \aleph_0$, which is countable.
    \item For the real number set, $\card(\R) = \aleph_1 > \aleph_0$.
\end{itemize}
The \emph{power set} of a set $\Omega$ is the set of all subsets, which is denoted by $2^\Omega$.
\begin{itemize}
    \item For $\Z$, $\card(2^\Z) = \aleph_1 = \card(\R)$.
\end{itemize}

In practical, sets in probability are: \textbf{finite, countable, reals} and \textbf{their variants}.

When we conduct an experiment, we want to know whether a subset occurs or not. For example, if we take a number from $\R$, the probability will be 0 because $\card(\R)$ is very large. Therefore,
\begin{center}
\emph{we are interested in $A \subset \Omega$ in probability, where $A$ is a collection of subsets in probability.}
\end{center}

\subsection{Events and fields}
\begin{definition}
The \textbf{events} are subsets of the sample space $\Omega$. 
\end{definition}

\begin{remark}
$\varnothing$ is called the \emph{impossible event}. The set $\Omega$ is called the \emph{certain event}. Events $A$ and $B$ are called \emph{disjoint} if their intersection is the empty set.
\end{remark}

We do \textbf{NOT} need all the subsets of $\Omega$ be events. It suffices for us to think of the collection of events as a subcollection $\mathcal{F}$ of the set of all subsets of $\Omega$.
\begin{definition}
Any collection $\mathcal{F}$ of subsets of $\Omega$ which satisfies the following three conditions is called a field:
\begin{enumerate}[(a)]
    \item if $A, B \in \mathcal{F}$, then $A \cup B \in \mathcal{F}$ and $A \cap B \in \mathcal{F}$ (actually $A\cap B \in \mathcal{F}$ is redundant); 
    \item if $A \in \mathcal{F}$, then $A^c \in \mathcal{F}$;
    \item the empty set $\varnothing$ belongs to $\mathcal{F}$.
\end{enumerate}
\end{definition}

\begin{remark}
Some implications from the properties of a filed $\mathcal{F}$:
\begin{itemize}
    \item if $A_1, \dots, A_n \in \mathcal{F}$, then $\bigcup_{i=1}^n A_i \in \mathcal{F}$;
    \item $\varnothing^C = \Omega \in \mathcal{F}$;
    \item $A \cap B = (A^c \cup B^c)^C \in \mathcal{F}$.
\end{itemize}
\end{remark}

\begin{question}
what if we have an $\Omega$ such that $\card(\Omega) = \aleph_0$? In this case, we may need $2^\Omega$ to study the probability. 
\end{question}

\begin{definition}
A collection $\mathcal{F}$ of subsets of $\Omega$ is called a $\sigma$-field if it satisfies the following conditions: 
\begin{enumerate}[(a)]
    \item $\varnothing \in \mathcal{F}$;
    \item if $A \in \mathcal{F}$, then $A^c \in \mathcal{F}$;
    \item if $A_1, A_2, \dots \in \mathcal{F}$, then $\bigcup_{i=1}^\infty A_i \in \mathcal{F}$.
\end{enumerate}
\end{definition}

\begin{remark}
$\sigma$-fields are closed under the operation of taking countable intersections.
\end{remark}

\begin{example}
The smallest $\sigma$-field associated with $\Omega$ is the collection $\mathcal{F} = \{\varnothing, \Omega \}$.
\end{example}

\begin{example}
If $A \subset \Omega$, then $\mathcal{F} = \{\varnothing, A, A^c, \Omega \}$ is a $\sigma$-field. 
\end{example}

\begin{example}
The power set of $\Omega$ is a $\sigma$-field.
\end{example}

With any experiment we may associate a pair $\{\Omega, \mathcal{F}\}$, where $\Omega$ is the set of all possible outcomes or elementary events and $\mathcal{F}$ is a $\sigma$-field of subsets of $\Omega$ which contains \emph{all the events in whose occurrences we may be interested}; henceforth, to call a set $A$ an event is equivalent to asserting that $A$ belongs to the $\sigma$-field in question. 


\section{Probability}
Assume we have a ``repeatable" experiment and we repeat the experiment a large number $N$ of times. Let $A \in \Omega$ and $N(A)$ be the number of $A$ occurs in the $N$ trails. Intuitively, the ratio $N(A)/ N$ appears to converge to a constant limit as $N$ increases. In practice, we have 
\begin{itemize}
    \item $0 \leq N(A)/ N \leq 1$;
    \item if $A, B$ are disjoint, then $N(A \cup B) = N(A) + N(B)$. (finite additive and countably additive)
\end{itemize}

\begin{definition}
A \textbf{probability measure} $\Prob$ on $\{ \Omega, \mathcal{F} \}$ is a function $\Prob : \mathcal{F} \to [0,1]$ satisfying
\begin{enumerate}[(a)]
    \item $\Prob(\varnothing) = 0$;
    \item $\Prob(\Omega) = 1$;
    \item  if $A_1, A2, \dots$ is a collection of disjoint members of $\mathcal{F}$, in that $A_i \cap A_j = \varnothing$ for all pairs $i, j$ satisfying $i \neq j$, then
    \begin{equation*}
        \Prob\left( \bigcup_{i=1}^\infty A_i \right) = \sum_{i=1}^\infty \Prob (A_i).
    \end{equation*}
\end{enumerate}
\end{definition}
The triple $\{ \Omega, \mathcal{F}, \Prob \}$, comprising a set $\Omega$, a $\sigma$-field $\mathcal{F}$ of subsets of $\Omega$, and a probability measure $\Prob$ on $\{\Omega, \mathcal{F}\}$, is called a \textbf{probability space}. We can associate a probability space $\{ \Omega, \mathcal{F}, \Prob \}$ with any experiment, and all questions associated with the experiment can be reformulated in terms of this space. 

\begin{remark}
A probability measure is a special example of what is called a \emph{measure} on the pair $\{\Omega, \mathcal{F}\}$. A measure is a function $\mu: \mathcal{F} \to [0, \infty)$ satisfying $\mu(\varnothing) = 0$ together with (c) above. A measure $\mu$ is a probability measure if $\mu(\Omega) = 1$. 
\end{remark}

\begin{lemma}
We can deduce some lemmas from the definition.
\begin{enumerate}[(a)]
    \item $\Prob(A^c) = 1 - \Prob(A)$;
    \item if $A \subseteq B$, then $\Prob(B) = \Prob(A) + \Prob(B\backslash A) \geq \Prob(A)$;
    \item $\Prob(A \cup B) = \Prob(A) + \Prob(B) - \Prob(A \cap B)$;
    \item more generally, if $A_1, A_2, \dots, A_n$ are events, then 
    \begin{equation*}
        \begin{split}
        \Prob\left( \bigcup_{i=1}^n A_i \right) =& \sum_{i}\Prob(A_i) - \sum_{i < j} \Prob(A_i \cap A_j) + \sum_{i<j<k} \Prob(A_i \cap A_j \cap A_k) - \cdots \\
        & +(-1)^{n+1} \Prob(A_1 \cap A_2 \cap \cdots \cap A_n)
        \end{split}
    \end{equation*}
\end{enumerate}
\end{lemma}

\begin{lemma}
Let $A_1, A_2, \dots$ be an increasing sequence of events, so that $A_1 \subset A_2 \subset A_3 \subset \cdots$, and write $A$ for their limit:
\begin{equation*}
    A = \bigcup_{i=1}^\infty A_i = \lim_{i \to \infty} A_i.
\end{equation*}
Then $\Prob(A) = \lim_{i\to\infty} \Prob(A_i)$. 

Similarly, if $B_1, B_2, \dots$ is a decreasing sequence of events, so that $B_1 \supseteq  B_2 \supseteq B_3 \supseteq \cdots$, then 
\begin{equation*}
    B = \bigcap_{i=1}^\infty B_i = \lim_{i \to \infty} B_i
\end{equation*}
satisfies $\Prob(B) = \lim_{i\to\infty} \Prob(B_i)$
\end{lemma}


\section{Some useful concepts}
\subsection{Conditional probability}
What if we only care how many times does $A$ occur only when $B$ occurs? $N(A\cap B) / N(B)$, the universe is changed. 
\begin{definition}
If $\Prob(B) > 0$, then the \textbf{conditional probability} that $A$ occurs given that $B$ occurs is defined to be
\begin{equation*}
    \Prob(A \spacevert B) = \frac{\Prob(A \cap B)}{\Prob(B)}.
\end{equation*}
\end{definition}

\begin{definition}
Suppose $\{B_i\}$ is a finite set. If $B_i$ are all disjoint and $\bigcup_{i=1}^n B_i = \Omega$, then $\{B_i\}$ is called a \textbf{partition} of $\Omega$.
\end{definition}

\begin{lemma}
For any events $A$ and $B$ such that $0 < \Prob(B) < 1$,
\begin{equation*}
    \Prob(A) = \Prob(A \spacevert B) \Prob(B) + \Prob(A \spacevert B^c) \Prob(B^c).
\end{equation*}
More generally, let $B_1,  B_2, \dots, B_n$ be a partition of $\Omega$ such that $\Prob(B_i) > O$ for all $i$. Then 
\begin{equation*}
    \Prob(A) = \sum_i \Prob(A \spacevert B_i) \Prob(B_i).
\end{equation*}
\end{lemma}

\subsection{Independence}
Intuition: an event $A$ occurs doesn't affect the probability of $A$ occurs when $B$ occurs, which means $\Prob(A) = \Prob(A\spacevert B)$.

\begin{definition}
Events $A$ and $B$ are called \textbf{independent} if 
\begin{equation*}
    \Prob(A \cap B) = \Prob(A) \Prob(B).
\end{equation*}
More generally, a family $\{A_i \spacevert i \in I\}$ is called independent if
\begin{equation*}
    \Prob \left( \bigcap_{i \in J} A_i \right) = \prod_{i\in J} \Prob(A_i)
\end{equation*}
for all finite subsets $J$ of $I$. 
\end{definition}

\begin{caution}
A common \emph{student error} is to make the fallacious statement that $A$ and $B$ are independent if $A \cap B = \varnothing$.
\end{caution}

\begin{remark}
If the family $\{A_i \spacevert i \in I\}$ has the property that 
for all 
\begin{equation*}
    \Prob(A_i \cap A_j) = \Prob(A_i) \Prob(A_j) \quad \forall i\neq j, 
\end{equation*}
then it is called \emph{pairwise independent}. \textbf{Pairwise-independent families are not necessarily independent.} 
\end{remark}

\section{Completeness and product space}
\begin{lemma}
If $\mathcal{F}$ and $\mathcal{G}$ are two $\sigma$-fields of subsets of $\Omega$, then their intersection $\mathcal{F} \cap \mathcal{G}$ is a $\sigma$-field also. More generally, if $\{ \mathcal{F}_i \spacevert  i \in I \}$ is a family of $\sigma$-fields of subsets of $\Omega$, then $\mathcal{G} = \bigcap_{i\in I} \mathcal{F}_i $ is a $\sigma$-field also. 
\end{lemma}

\begin{newnotion}{Completeness}
Let $\{\Omega, \mathcal{F}, \Prob\}$ be a probability space. Any event $A$ which has zero probability, that is $\Prob(A) = 0$, is called \emph{null}. It may seem reasonable to suppose that any subset $B$ of a null set $A$ will itself be null, but this may be without meaning since $B$ may not be an event, and thus $\Prob(B)$ may not be defined. 
\end{newnotion}

\begin{definition}
A probability space $\{\Omega, \mathcal{F}, \Prob\}$ is called \textbf{complete} if all subsets of null sets are events. 
\end{definition}

Any incomplete space can be completed thus. Let $\mathcal{N}$ be the collection of all subsets of null sets in $\mathcal{F}$ and let $\mathcal{G} = \sigma (\mathcal{F} \cup \mathcal{N})$ be the smallest $\sigma$-field which contains all sets in $\mathcal{F}$ 
and $\mathcal{N}$. It can be shown that the domain of $\Prob$ may be extended in an obvious way from $\mathcal{F}$ to $\mathcal{G}$; $\{\Omega, \mathcal{G}, \Prob\}$ is called the completion of $\{\Omega, \mathcal{F}, \Prob\}$. 

\begin{newnotion}{Product space}
Suppose two experiments have associated probability spaces $\{ \Omega_1, \mathcal{F}_1, \Prob_1 \}$ and  $\{ \Omega_2, \mathcal{F}_2, \Prob_2 \}$ respectively. The sample space of the pair of experiments, considered jointly, is the collection $\Omega_1 \times \Omega_2 = \{ (\omega_1 , \omega_2) \spacevert w_1 \in \Omega_1, \omega_2 \in \Omega_2 \}$ of ordered pairs. The appropriate $\sigma$-field of events is \textbf{more complicated} to construct. \textbf{The family of all such sets, $\mathcal{F}_1 \times \mathcal{F}_2 = \{A_1 \times A_2 \spacevert A_1 \in \mathcal{F}_1 , A_1 \in \mathcal{F}_2 \}$, is NOT in general a $\sigma$-field.}
\end{newnotion}

\begin{remark}
There exists a unique smallest $\sigma$-field $\mathcal{G} = \sigma(\mathcal{F}_1 \times \mathcal{F}_2)$ of subsets of $\Omega_1 \times \Omega_2$ which contains $\mathcal{F}_1 \times \mathcal{F}_2$. All we require now is a suitable probability function on $( \Omega1 \times \Omega_2, \mathcal{G} )$. Let $\Prob_{12}: \mathcal{F}_1 \times \mathcal{F}_2 \to [0, 1]$ be given by: 
\begin{equation*}
    \Prob(A_1 \times A_2) = \Prob_1(A_1) \times \Prob_2(A_2) \quad \text{for $A_1 \in \mathcal{F}_1, A_2 \in \mathcal{F}_2$}.
\end{equation*}
It can be shown that the domain of $\Prob_{12}$ can be extended from $\mathcal{F}_1 \times \mathcal{F}_2$ to the whole of $\mathcal{G} = \sigma(\mathcal{F}_1 \times \mathcal{F}_2)$.
\end{remark}

\begin{definition}
The probability space $(\mathcal{F}_1 \times \mathcal{F}_2, \mathcal{G}, \Prob_{12})$ is called the \textbf{product space} of $(\Omega_1, \mathcal{F}_1, \Prob_1)$ and  $(\Omega_2, \mathcal{F}_2, \Prob_2)$. The measure $\Prob_{12}$ is sometimes called the `product measure'. 
\end{definition}


\end{document}
