\documentclass[11pt]{article}
\usepackage[utf8]{inputenc}
\usepackage[shortlabels]{enumitem}
\usepackage{amsmath}
\usepackage{amssymb}
\usepackage{amsthm}
\usepackage{xcolor}
\usepackage{graphicx}
\usepackage{blkarray}

\usepackage{hyperref}
\hypersetup{
    colorlinks=true,
    linkcolor=blue,
    filecolor=magenta,      
    urlcolor=cyan,
}


% setting page format
\topmargin -.5in
\textheight 9in
\oddsidemargin -.25in
\evensidemargin -.25in
\textwidth 7in
\setlength{\parindent}{0 in}
\setlength{\parskip}{0.1 in}

% setting new environments
\newtheorem*{theorem}{Theorem}
\newtheorem*{lemma}{Lemma}
\newtheorem*{corollary}{Corollary}

\theoremstyle{definition}
\newtheorem*{definition}{Definition}
\newtheorem*{example}{Example}
\newtheorem*{problem}{Problem}
\newtheorem*{question}{Question}

\theoremstyle{remark}
\newtheorem*{remark}{Remark}

\newenvironment{solution}[1][Solution]{\textbf{#1:} \par}{\ $\blacksquare$}

\newenvironment{newnotion}[1]{\textbf{#1.}}

\newenvironment{caution}{$\bigstar\bigstar\bigstar$\textbf{Caution.}}

%\newenvironment{proof}[1][Proof]{\textbf{#1:} \par}{\ \rule{0.5em}{0.5em}}
%\newenvironment{problem}[1]{\textbf{#1:} }

% define new commands
\renewcommand{\hat}{\widehat}
\renewcommand{\tilde}{\widetilde}

\newcommand{\numpy}{{\tt numpy}}    % tt font for numpy
\newcommand{\dom}[1]{\mathbf dom #1}
\newcommand{\tr}{\mathbf{tr}}
\newcommand{\sgn}{\text{sgn}}
\newcommand{\spacevert}{\;\vert\;}
\newcommand{\ie}{i.e.}
\newcommand{\eg}{e.g.}

\newcommand{\N}{\mathbb{N}}
\newcommand{\E}{\mathbb{{E}}}
\newcommand{\Q}{\mathbb{Q}}
\newcommand{\R}{\mathbb{R}}
\newcommand{\C}{\mathbb{C}}
\newcommand{\Z}{\mathbb{Z}}
\newcommand{\mS}{\mathbb{S}}

% for probability
\newcommand{\Exp}{\mathbf{E}}
\newcommand{\Var}{\mathbf{Var}}
\newcommand{\Prob}{\mathbf{P}}
\newcommand{\card}{\mathbf{card}}
\newcommand{\cov}{\mathbf{cov}}
\newcommand{\corr}{\mathbf{corr}}
\newcommand{\1}{\mathbf{1}}


% for optimizations
\newcommand{\subto}{\text{s.t.}}

% for norms
\newcommand{\norm}[1]{\left\Vert #1 \right\Vert}
\newcommand{\abs}[1]{\left\vert #1 \right\vert}

\begin{document}
% ========== Edit your name here
\title{MATH 2901 Basic Probability Lecture Notes 2}
\author{Instructor: Richard Kleeman}
\date{}
\maketitle

%\medskip

% ========== Contents begin here ==============
## Random variables
\begin{definition}
**A random variable** is a function $X: \Omega \to \mathbb{R}$ with the property that $\{ \omega\in \Omega \;\vert\; X(\omega) \leq x\} \in \mathcal{F}$ for each $x \in \mathbb{R}$. Such a function is said to be $\mathbf{\mathcal{F}}$**-measurable**. 
\end{definition}

\begin{example}
Tossing two dice. We define 
\begin{equation*}
    X = \begin{cases} 1 & \text{get double ($i=j$)} \\ 0 & \text{not double $(i\neq j)$} \end{cases}.
\end{equation*}
Then $X = (i,j) \in \mathbb{R}$.
\end{example}

\begin{remark}
We are interested in $g(x) = \mathbf{P}\{ \omega \;\vert\; X(\omega) = x \}$. But sometimes it doesn't work very well. Probability triplet is defined as $\{ \Omega, \mathcal{F}, \mathbf{P} \}$. If $\Omega$ is countable, then $g(x)$ is okay. But if $\Omega$ is uncountable like intervals, then $\mathbf{P}(X=x)$ doesn't make much sense because the cardinality is too large.
\end{remark}

\begin{definition}
The **distribution function** of a random variable $X$ is the function $F: \mathbb{R} \to 
[0, 1]$ given by $F(x) = \mathbf{P}(X \leq x)$. 
\end{definition}

\begin{example}
The distribution function of preceeding example is 
\begin{equation*}
    F(x) = \begin{cases} 0 & x \leq 0, \\ \frac{30}{36} & 0 < x \leq 1, \\ 1 & x > 1. \end{cases}
\end{equation*}
\end{example}

Notice that $\{ \omega \;\vert\; X(\omega) \leq x\}$ defines an event. It is an element in the corresponding $\sigma$-field. Denote $A(x) = \{ \omega \;\vert\; X(\omega) \leq x \}$. Along with $A(x)$, we can define
\begin{equation*}
    A^c(x) = \{ \omega \;\vert\; X(\omega) > x \}, \quad A(x, y) = A^c(x) \cap A(y) = \{ \omega \;\vert\; x < X(\omega) \leq y \}.
\end{equation*}
Tow points worth noting:
\begin{enumerate}[(a)]
    \item $F$ must be defined for **all** $x \in \mathbb{R}$.
    \item $A(x)$ should **belongs to** $\mathcal{F}$. Otherwise, we cannot talk about the probability of $\mathbf{P}(A(x))$. Then the definition of distribution function is meaningless.
\end{enumerate}

\begin{lemma}
A distribution function $F$ has the following properties:
\begin{enumerate}[(a)]
    \item $\lim_{x \to -\infty}F(x) = 0$, $\lim_{x\to\infty} F(x) = 1$,
    \item if $x < y$, then $F(x) \leq F(y)$,
    \item $F$ is right-continuous, that is $F(x+h) \to F(x)$ as $h \downarrow 0$. (left-continuous is not necessary)
\end{enumerate}
\end{lemma}

\begin{example}
**Indicator functions.** A particular class of Bernoulli variables is very useful in probability theory. Let $A$ be an event and let $I_A : \Omega \to \mathbb{R}$ be the indicator function of $A$; that is,
\begin{equation*}
    I_A(\omega) = \begin{cases} 1 & \text{if $\omega \in A$}, \\ 0 & \text{if $\omega \in A^c$}. \end{cases}
\end{equation*}
Then $I_A$ is a Bernoulli random variable taking the values 1 and 0 with probabilities $\mathbf{P}(A)$ and $\mathbf{P}(A^c)$ respectively. Suppose $\{B_i \;\vert\; i \in I\}$ is a family of disjoint events with $A \subseteq \bigcup_{i\in I} B_i$. Then 
\begin{equation*}
    I_A = \sum_{i} = I_{A\cap B_i},
\end{equation*}
an identity which is often useful.
\end{example}

\begin{lemma}
Let $F$ be the distribution function of $X$. Then
\begin{enumerate}[(a)]
    \item $\mathbf{P}(X>x) = 1-F(x)$, 
    \item $\mathbf{P}(x < X \leq y) = F(y) - F(x)$,
    \item $\mathbf{P}(X = x) = F(x) - \lim_{y \uparrow x} F(y)$.
\end{enumerate}
\end{lemma}

A random variable $X$ with distribution function $F$ is said to have two ``tails" given by 
\begin{equation*}
    T_1 (X) = \mathbf{P}(X > x) = 1 - F(x), \quad T_2(X) = \mathbf{P}(X \leq -x) = F(-x),    
\end{equation*}
where $x$ is large and positive. The rates at which the $T_i$ decay to zero as $x\to\infty$ have a substantial effect on the existence or non-existence of certain associated quantities called the ``moments" of the distribution. 


## Different random variables
### Discrete random variables
\begin{definition}
The random variable $X$ is called **discrete** if it takes values in some **countable** subset $\{ x_1, x_2, \dots \}$, only, of $\mathbb{R}$. The discrete random variable $X$ has **(probability) mass function** $f : \mathbb{R} \to [0, 1]$ given by $f(x) = \mathbf{P}(X = x)$. 
\end{definition}
We shall see that the distribution function of a discrete variable has jump discontinuities 
at the values $x_1 , x_2, \dots$ and is constant in between; such a distribution is called \emph{atomic}.

### Continuous random variables
\begin{definition}
The random variable $X$ is called **continuous** if its distribution function can 
be expressed as 
\begin{equation*}
    F(x) = \int_{-\infty}^x f(u) du \qquad x \in \mathbb{R}
\end{equation*}
for some **integrable** function $f : \mathbb{R} \to [0, \infty)$ called the **(probability) density function** of $X$. 
\end{definition}
Some point worth noting:
\begin{enumerate}[(a)]
    \item The density function $f(x)$ is not unique. We can add some separate points to $f$, and it doesn't affect the integration.
    \item $F$ must be **absolutely continuous**. This implies $F$ is continuous. We can also deduce that the probability at certain point must be zero. \emph{i.e.}, $\mathbf{P}(X=x) = 0$.
\end{enumerate}

There is another sort of random variable, called ``singular".


## Random vectors
### Definition
The random vector is a function $X : \Omega \to \mathbb{R}^n$. For example, $\mathbf{X} = (X, Y)$ for $n=2$. We can also define the distribution function for such $X$. But we need to first introduce the ordering in $\mathbb{R}^n$.
\begin{center}
    By definition, $(x_1, y_1) < (x_2, y_2)$ if and only if $x_1 < x_2$  **AND**  $y_1 < y_2$.
\end{center}

\begin{definition}
The **joint distribution function** of a random vector $\mathbf{X} = (X_1, X_2 \dots, X_n)$ on the probability space $\{ \Omega, \mathcal{F}, \mathbf{P} \}$ is the function $F_{\mathbf{X}} : \mathbb{R}^n \to [0, 1]$ given by $F_{\mathbf{X}}(x) = \mathbf{P}(\mathbf{X} \leq \mathbf{x})$ for $\mathbf{x} \in \mathbb{R}^n$. 
\end{definition}

\begin{remark}
The joint probability $\mathbf{P}(\mathbf{X} \leq \mathbf{x}) = \mathbf{P}(X_1 \leq x_1, \dots, X_n \leq x_n)$. $\{\mathbf{X} \leq \mathbf{x} \}$ is an abbreviation for the event $\{\omega \in \Omega \;\vert\; \mathbf{X}(w) \leq \mathbf{x}\}$.
\end{remark}

\begin{lemma}
The joint distribution function $F_{X, Y}$ of the random vector $(X, Y)$ has the following properties: 
\begin{enumerate}[(a)]
    \item $\lim_{x, y\to -\infty} F_{X, Y}(x, y) = 0, \lim_{x,y\to\infty} F_{X, Y}(x,y) = 1$,
    \item if $(x_1, y_1) \leq (x_2, y_2)$, then $F_{X,Y}(x_1, y_1) \leq F_{X, Y}(x_2, y_2)$,
    \item $F_{X,Y}$ is continuous from above, in that 
    \begin{equation*}
        F_{X,Y}(x+u, y+v) \to F_{X,Y}(x,y) \quad as \quad u, v \downarrow 0.
    \end{equation*}
\end{enumerate}
\end{lemma}

### Marginalization 
\begin{gather*}
    \lim_{y\to\infty} F_{X,Y} = F_X(x) = \mathbf{P}(X \leq x), \\
    \lim_{x\to\infty} F_{X,Y} = F_Y(y) = \mathbf{P}(Y \leq y).
\end{gather*}
The functions $F_X$ and $F_Y$ are called the ``marginal" distribution functions of $F_{X,Y}$. $F_{X,Y}$ can determine two marginals $F_X$ and $F_Y$, but **converse is NOT true**.

### Discrete and continuous distribution
\begin{definition}
The random variables $X$ and $Y$ on the probability space $\{ \Omega, \mathcal{F}, \mathbf{P} \}$ are called **(jointly) discrete** if the vector $(X, Y)$ takes values in some **countable** subset of $\mathbb{R}^2$ only. The jointly discrete random variables $X, Y$ have **joint (probability) mass function** $f : \mathbb{R}^2 \to [0,1]$ given by $f(x, y) = \mathbf{P}(X = x, Y = y)$. 
\end{definition}

\begin{definition}
The random variables $X$ and $Y$ on the probability space $\{ \Omega, \mathcal{F}, \mathbf{P} \}$ are called **(jointly) continuous** if their joint distribution function can be expressed as 
\begin{equation*}
    F_{X,Y}(x,y) = \int_{u = -\infty}^x \int_{v = -\infty}^y f(u,v)du dv \qquad x, y\in\mathbb{R},
\end{equation*}
for some **integrable** function $f : \mathbb{R}^2 \to [0, \infty)$ called the **joint (probability) density function** of the pair $(X, Y)$. 
\end{definition}

\end{document}