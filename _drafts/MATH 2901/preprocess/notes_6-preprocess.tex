\documentclass[11pt]{article}
\usepackage[utf8]{inputenc}
\usepackage[shortlabels]{enumitem}
\usepackage{amsmath}
\usepackage{amssymb}
\usepackage{amsthm}
\usepackage{xcolor}
\usepackage{graphicx}
\usepackage{blkarray}

\usepackage{hyperref}
\hypersetup{
    colorlinks=true,
    linkcolor=blue,
    filecolor=magenta,      
    urlcolor=cyan,
}


% setting page format
\topmargin -.5in
\textheight 9in
\oddsidemargin -.25in
\evensidemargin -.25in
\textwidth 7in
\setlength{\parindent}{0 in}
\setlength{\parskip}{0.1 in}

% setting new environments
\newtheorem*{theorem}{Theorem}
\newtheorem*{lemma}{Lemma}
\newtheorem*{corollary}{Corollary}

\theoremstyle{definition}
\newtheorem*{definition}{Definition}
\newtheorem*{example}{Example}
\newtheorem*{problem}{Problem}
\newtheorem*{question}{Question}

\theoremstyle{remark}
\newtheorem*{remark}{Remark}

\newenvironment{solution}[1][Solution]{\textbf{#1:} \par}{\ $\blacksquare$}

\newenvironment{newnotion}[1]{\textbf{#1.}}

\newenvironment{caution}{$\bigstar\bigstar\bigstar$\textbf{Caution.}}

%\newenvironment{proof}[1][Proof]{\textbf{#1:} \par}{\ \rule{0.5em}{0.5em}}
%\newenvironment{problem}[1]{\textbf{#1:} }

% define new commands
\renewcommand{\hat}{\widehat}
\renewcommand{\tilde}{\widetilde}

\newcommand{\numpy}{{\tt numpy}}    % tt font for numpy
\newcommand{\dom}[1]{\mathbf dom #1}
\newcommand{\tr}{\mathbf{tr}}
\newcommand{\sgn}{\text{sgn}}
\newcommand{\spacevert}{\;\vert\;}
\newcommand{\ie}{i.e.}
\newcommand{\eg}{e.g.}

\newcommand{\N}{\mathbb{N}}
\newcommand{\E}{\mathbb{{E}}}
\newcommand{\Q}{\mathbb{Q}}
\newcommand{\R}{\mathbb{R}}
\newcommand{\C}{\mathbb{C}}
\newcommand{\Z}{\mathbb{Z}}
\newcommand{\mS}{\mathbb{S}}

% for probability
\newcommand{\Exp}{\mathbf{E}}
\newcommand{\Var}{\mathbf{Var}}
\newcommand{\Prob}{\mathbf{P}}
\newcommand{\card}{\mathbf{card}}
\newcommand{\cov}{\mathbf{cov}}
\newcommand{\corr}{\mathbf{corr}}
\newcommand{\1}{\mathbf{1}}


% for optimizations
\newcommand{\subto}{\text{s.t.}}

% for norms
\newcommand{\norm}[1]{\left\Vert #1 \right\Vert}
\newcommand{\abs}[1]{\left\vert #1 \right\vert}

\begin{document}
% ========== Edit your name here
\title{MATH 2901 Basic Probability Lecture Notes 6}
\author{Instructor: Richard Kleeman}
\date{}
\maketitle

%\medskip

% ========== Contents begin here ==============
This lecture we study the expectation of the average of $n$ i.i.d. random variables $X_i$, $i=1,\dots, n$, i.e., the expectation of $Q_n = \frac{1}{n} (X_1 + \dots + X_n)$. As $n \to \infty$, we will introduce the central limit theorem and show that $Q_n$ converges to a normal distribution provided $\mathbf{Var}(X_i)$ exists.


## Sums of Discrete Random Variables
Suppose $X$ and $Y$ are two independent discrete random variables with distribution functions $p_X(x)$ and $p_Y(y)$. Let $Z = X + Y$, we want to find the the distribution function of $Z$. 

Suppose that $X = k$, where $k$ is some integer. Then $Z = z$ if and only if $Y = z-k$. So the event $Z = z$ is the union of the pairwise disjoint events
$$\begin{equation}
    (X=k) \quad  \text { and } \quad  (Y=z-k)
\end{equation}$$
where $k$ runs over the integers. Since these events are pairwise disjoint, we have
$$$$\begin{equation}
    \label{eq:6.1}
    \tag{6-1}
    \mathbf{P}(Z=z)=\sum_{k=-\infty}^{\infty} \mathbf{P}(X=k)  \mathbf{P}(Y=z-k)
\end{equation}$$$$
which is the **distribution function of the random variable $Z$**.

<div class="theorem-block"><p><b>Definition.</b> 
Let $X$ and $Y$ be two independent integer-valued random variables, with distribution functions $p_X(x)$ and $p_Y(y)$ respectively. Then the **convolution** of $p_X(x)$ and $p_Y(y)$ is the distribution function $p_Z = p_X * p_Y$ given by
$$\begin{equation}
    p_Z(z) = \sum_{x} p_X(x) p_Y(z-x)
\end{equation}$$
for $x\in \mathbb{Z}$. The function $p_Z(z)$ is the distribution function of the random variable $Z = X + Y$.
</p></div>


## Sums of Continuous Random Variables
<div class="theorem-block"><p><b>Definition.</b> 
Let $X$ and $Y$ be two continuous random variables with density functions $f_X(x)$ and $f_Y(y)$ respectively. Assume that both $f_X(x)$ and $f_Y(y)$ are defined for all real numbers. Then the **convolution** $f * g$ of $f$ and $g$ is the function given by
$$$$\begin{equation}
    \label{eq:6.2}
    \tag{6-2}
    (f_X * f_Y)(z) = \int_{-\infty}^{+\infty} f_X(z-y) f_Y(y) d y = \int_{-\infty}^{+\infty} f_Y(z-x) f_X(x) d x.
\end{equation}$$$$
</p></div>

<div class="theorem-block"><p><b>Theorem.</b> 
Let $X$ and $Y$ be two independent random variables with density functions $f_X(x)$ and $f_Y(y)$ defined for all $x$ and $y$. Then the sum $Z = X + Y$ is a random variable with density function $f_Z(z)$, where $f_Z$ is the convolution of $f_X$ and $f_Y$.
</p></div>

<div class="theorem-block"><p><b>Theorem.</b> 
Let $\{X_i\}$, $i=1,\dots, n$ be a sequence of independent random variables with density functions $f_{X_1}(x), \dots, f_{X_n}(x)$ respectively, then we have 
$$\begin{equation}
    f_{X_1 + \cdots + X_n}(x) = f_{X_1} * \left( f_{X_2} * ( \cdots * f_{X_n}) \right)(x).
\end{equation}$$
</p></div>

<div class="remark-block"><p><b>Example.</b> [Sum of two independent uniform random variables]
Let $X$ and $Y$ be random variables describing our choices and $Z = X + Y$ their sum. Then we have
$$\begin{equation}
    f_{X}(x)=f_{Y}(x)=\left\{\begin{array}{ll}{1} & {\text { if } 0 \leq x \leq 1} \\ {0} & {\text { otherwise }}\end{array}\right.
\end{equation}$$
and the density function for the sum is given by
$$\begin{equation}
    f_{Z}(z)=\int_{-\infty}^{+\infty} f_{X}(z-y) f_{Y}(y) d y = \int_{0}^{1} f_{X}(z-y) d y.
\end{equation}$$
Now the integrand is $0$ unless $0 \leq z-y \leq 1$ and then it is $1$. So if $0 \leq z \leq 1$, we have 
$$\begin{equation}
    f_{Z}(z)=\int_{0}^{z} d y=z,
\end{equation}$$
while if $1 < z \leq 2$, we have
$$\begin{equation}
    f_{Z}(z)=\int_{z-1}^{1} d y=2-z, 
\end{equation}$$
and if $z < 0$ or $z > 2$ we have $f_Z(z) = 0$. Hence
$$\begin{equation}
    f_{Z}(z)=\left\{\begin{array}{ll}{z,} & {\text { if } 0 \leq z \leq 1} \\ {2-z,} & {\text { if } 1<z \leq 2} \\ {0,} & {\text { otherwise }}\end{array}\right.
\end{equation}$$
</p></div>

<div class="remark-block"><p><b>Example.</b> [Sum of two independent exponential random variables]
Let $X, Y$, and $Z = X + Y$ denote the relevant random variables, and $f_X$ , $f_Y$ , and $f_Z$ their densities. Then
$$\begin{equation}
    f_{X}(x)=f_{Y}(x)=\left\{\begin{array}{ll}{\lambda e^{-\lambda x},} & {\text { if } x \geq 0} \\ {0,} & {\text { otherwise }}\end{array}\right.
\end{equation}$$
If $z > 0$,
$$\begin{equation}
    f_{Z}(z)=\int_{-\infty}^{+\infty} f_{X}(z-y) f_{Y}(y) d y =\int_{0}^{z} \lambda e^{-\lambda(z-y)} \lambda e^{-\lambda y} d y =\int_{0}^{z} \lambda^{2} e^{-\lambda z} d y =\lambda^{2} z e^{-\lambda z},
\end{equation}$$
while if $z < 0$, $f_Z(z) = 0$. Hence 
$$\begin{equation}
    f_{Z}(z)=\left\{\begin{array}{ll}{\lambda^{2} z e^{-\lambda z},} & {\text { if } z \geq 0} \\ {0,} & {\text { otherwise }}\end{array}\right.
\end{equation}$$
</p></div>


## Generating functions
A sequence $a = \{a_i \;\vert\; i = 0, 1 , 2, \dots \}$ of real numbers may contain a lot of information. One concise way of storing this information is to wrap up the numbers together in a ``generating function". For example, the (ordinary) **generating function** of the sequence $a$ is the function $G_a$ defined by 
$$\begin{equation}
    G_{a}(s)=\sum_{i=0}^{\infty} a_{i} s^{i} \quad \text { for } s \in \mathbb{R} \text { for which the sum converges. }
\end{equation}$$
In many circumstances it is easier to work with the generating function $G_a$ than with the original sequence $a$.

<div class="theorem-block"><p><b>Theorem.</b> [Abel's theorem]
If $a_i \geq 0$ for all $i$ and $G_a(s)$ is finite for $\left\vert s \right\vert < 1$, then $\lim_{s \uparrow 1} G_a (s) = \sum_{i=1}^ \infty a_i$, whether the sum is finite or equals $+\infty$. This standard result is useful when the radius of convergence $R$ satisfies $R = 1$, since then one has no a priori right to take the limit as $s \uparrow 1$.
</p></div>


### Moment generating function
<div class="theorem-block"><p><b>Definition.</b> 
The **moment generating function** of the random variable $X$ is the function $M: \mathbb{R} \mapsto [0, \infty)$ given by the Laplace transform of the corresponding p.d.f. $f_X(s)$:
$$\begin{equation}
    M_X(t) = \mathbf{E} \left( e^{tX} \right) = \int_{-\infty}^\infty e^{tx} f_X(x) dx,
\end{equation}$$
or corresponding p.m.f. $p_X(k)$:
$$\begin{equation}
    M_X(t) = \sum_{k} e^{t k} \mathbf{P}(X=k) 
    = \sum_k \sum_{n=0}^{\infty} \frac{(t k)^{n}}{n !} \mathbf{P}(X=k)
    = \sum_{n=0}^{\infty} \frac{t^{n}}{n !}\left(\sum_k k^{n} \mathbf{P}(X=k)\right) 
    = \sum_{n=0}^{\infty} \frac{t^{n}}{n !} \mathbf{E}\left(X^{n}\right).
\end{equation}$$
$M_X(-t)$ is so called **bilateral** Laplace transform of $f_X(x)$ or $p_X(k)$.
</p></div>

Under the assumption that $M_X(t)$ is infinitely differentiable at $t=0$, the following statements are true. 
<ol>[(a)]
    <li> $M^\prime(0) = \mathbf{E}(0) = \mu$.
    <li> $M^{(n)}(0) = \mathbf{E}(X^n)$.
    <li> Using Taylor's theorem, $M_X(t) = \sum_{k=0}^\infty \frac{t^k}{k!} \mathbf{E}(X^k)$.
</ol>

<div class="theorem-block"><p><b>Theorem.</b> 
If $X$ and $Y$ are independent, then
$$\begin{equation}
    \begin{split}
        M_{X+Y}(t) &= \int_{-\infty}^\infty e^{tz} f_{x+y}(z) dz \\
        &= \int_{-\infty}^\infty e^{tz} \int_{-\infty}^\infty f_X(x) f_Y(z-x) dx dz \\
        &= \int_{-\infty}^\infty e^{t(x+y)} \int_{-\infty}^\infty f_X(x) f_Y(y) dx dy \\
        &= M_X(t) M_Y(t).
    \end{split}
\end{equation}$$
</p></div>

### Characteristic functions
Sometimes $\mathbf{E}(e^{tX})$ may blow up. So we consider some transformations in the complex domain, which usually perform better.

<div class="theorem-block"><p><b>Definition.</b> 
The **characteristic function** of $X$ is the function $\phi: \mathbb{R} \mapsto \mathbb{C}$ defined by
$$\begin{equation}
    \phi(t)=\mathbf{E}\left(e^{i t X}\right) \quad \text { where } \quad i=\sqrt{-1}.
\end{equation}$$
We often write $\phi_x$ for the characteristic function of the random variable $X$. Characteristic functions are related to Fourier transforms. 
</p></div>

<div class="theorem-block"><p><b>Theorem.</b> 
The characteristic function $\phi$ satisfies:
<ol>[(a)]
    <li> $\phi(0) = 1$, $\left\vert \phi(t) \right\vert \leq 1$ for all $t$.
    <li> $\phi$ is uniformly continuous on $\mathbb{R}$ w.r.t. $t$.
    <li> If $X \sim \mathcal{N}(0,1)$, then $\phi_{X}(t) = e^{-t^2/2}$.
</ol>
</p></div>

\begin{proof}
We only prove the first statement.
$$\begin{equation}
    \begin{split}
        \left\vert \phi(t) \right\vert &= \left\vert \int_{-\infty}^\infty e^{itx} f(x)dx  \right\vert \\
        &\leq \int_{-\infty}^\infty \left\vert e^{itx} \right\vert f(x) dx \quad \text{(triangle inequality)} \\
        &= \int_{-\infty}^\infty f(x) dx \quad (\left\vert e^{itx} \right\vert=1) \\
        &= 1
    \end{split}
\end{equation}$$
\end{proof}

<div class="remark-block"><p><b>Example.</b> [Cauchy distribution]
If $f(x) = 
\frac{1}{\pi(1+x^2)}$, then the corresponding characteristic function is 
$$\begin{equation}
    \phi(t)=\frac{1}{\pi} \int_{-\infty}^{\infty} \frac{e^{i t x}}{1+x^{2}} d x = e^{-\left\vert t \right\vert}.
\end{equation}$$
</p></div>

<div class="theorem-block"><p><b>Theorem.</b> 
The following statements are true.
<ol>[(a)]
    <li> If $\phi^{(k)}(0)$ exists, then 
    $$\begin{equation}
        \left\{\begin{array}{ll}{\mathbf{E}\left|X^{k}\right|<\infty} & {\text { if $k$}  \text { is even }} \\ {\mathbf{E}\left|X^{k-1}\right|<\infty} & {\text { if $k$ } \text { is odd }}\end{array}\right.
    \end{equation}$$
    
    <li> If $\mathbf{E}(\left\vert X^k \right\vert) < \infty$, then 
    $$\begin{equation}
        \phi(t)=\sum_{j=0}^{k} \frac{\mathbb{E}\left(X^{j}\right)}{j !}(i t)^{j}+\mathrm{o}\left(t^{k}\right)
    \end{equation}$$
    and so $\phi^{(k)}(0)=i^{k} \mathbb{E}\left(X^{k}\right)$.
</ol>
</p></div>
<div class="theorem-block"><p><b>Theorem.</b> 
If $X$ and $Y$ are independent then 
$$\begin{equation}
    \phi_{X+Y}(t)=\phi_{X}(t) \phi_{Y}(t).
\end{equation}$$
Similarly, if $X_1, \dots, X_n$ are independent, then 
$$\begin{equation}
    \phi_{X_1 + \cdots + X_n}(t)= \prod_{i=1}^n \phi_{X_i}(t).
\end{equation}$$
</p></div>

<div class="theorem-block"><p><b>Theorem.</b> 
If $a, b \in \mathbb{R}$ and $Y = aX+b$, then $\phi_{Y}(t)=e^{i t b} \phi_{X}(a t)$.
</p></div>
\begin{proof}
$$\begin{equation}
    \phi_{Y}(t)=\mathbf{E}\left(e^{i t(a X+b)}\right)=\mathbf{E}\left(e^{i t b} e^{i(a t) X}\right) = e^{i t b} \mathbf{E}\left(e^{i(a t) X}\right)=e^{i t b} \phi_{X}(a t).
\end{equation}$$
\end{proof}

<div class="theorem-block"><p><b>Theorem.</b> 
Random variables $X$ and $Y$ are independent if and only if
$$\begin{equation}
    \phi_{X, Y}(s, t)=\phi_{X}(s) \phi_{Y}(t) \quad \text { for all } s \text { and } t.
\end{equation}$$
</p></div>

<div class="theorem-block"><p><b>Definition.</b> 
We say that the sequence $F_1 , F_2, \dots $ of distribution functions converges to the distribution function $F$, written $F_n \to F$, if $F(x) = \lim_{n\to\infty} F_n(x)$ at each point $x$ where $F$ is continuous.
</p></div>

<div class="theorem-block"><p><b>Theorem.</b> [Continnity theorem]
Suppose that $F_1 , F_2, \dots $ is a sequence of distribution functions 
with corresponding characteristic functions $\phi_1 , \phi_2, \dots $.
<ol>[(a)]
    <li> If $F_n \to F$ for some distribution function $F$ with characteristic function $\phi$, then $\phi_n(t) \to \phi(t)$ for all $t$.
    <li> Conversely, if $\phi(t) \lim_{n\to\infty} \phi_n(t)$ exists and is continuous at $t=0$, then $\phi$ is the characteristic function of some distribution function $F$, and $F_n \to F$.
</ol>
</p></div>


## Central limit theorem
<div class="theorem-block"><p><b>Definition.</b> 
If $X, X_1 , X_2 , \dots$ is a sequence of random variables with respective distribution functions $F, F_1, F_2, \cdots$, we say that $X_n$ converges in distribution to $X$, written $X_{n} \stackrel{\mathrm{D}}{\rightarrow} X$, if $F_n \to F$ as $n \to\infty$.
</p></div>

<div class="theorem-block"><p><b>Theorem.</b> [Central limit theorem]
Let $X_1 , X_2, \dots$ be a sequence of independent identically distributed random variables with finite mean $\mu$ and finite nonzero variance $\sigma^2$, and let $S_n = X_1 + X_2 + \cdots + X_n$. Then
$$\begin{equation}
    \frac{S_{n}-n \mu}{\sqrt{n \sigma^{2}}} \stackrel{\mathrm{D}}{\rightarrow} N(0,1) \quad \text { as } \quad n \rightarrow \infty.
\end{equation}$$
</p></div>

\begin{proof}
First, write $Y_i = \frac{X_i - \mu}{\sigma}$, and let $\phi_Y$ be the characteristic function of the $Y_i$. We have that 
$$\begin{equation}
    \phi_{Y}(t)=1-\frac{1}{2} t^{2}+o\left(t^{2}\right).
\end{equation}$$
Note that $Y_i$ are i.i.d. So the characteristic function of $\sum_{i=1}^n Y_i$ is
$$\begin{equation}
    \phi_n = [\phi_{Y}(t)]^n = \left[ 1-\frac{1}{2} t^{2}+o\left(t^{2}\right) \right]^n.
\end{equation}$$
Also, the characteristic function $\psi_n$ of
$$\begin{equation}
    U_{n}=\frac{S_{n}-n \mu}{\sqrt{n \sigma^{2}}}=\frac{1}{\sqrt{n}} \sum_{i=1}^{n} Y_{i}
\end{equation}$$
satisfies
$$\begin{equation}
    \psi_{n}(t)=\left\{\phi_{Y}(t / \sqrt{n})\right\}^{n}=\left\{1-\frac{t^{2}}{2 n}+o\left(\frac{t^{2}}{n}\right)\right\}^{n} \rightarrow e^{-\frac{1}{2} t^{2}} \quad \text { as } \quad  n \rightarrow \infty,
\end{equation}$$
where we used 
$$\begin{equation}
    \lim_{n\to\infty} \left( 1 + \frac{a}{n} \right)^n = e^a.
\end{equation}$$
The last function is the characteristic function of the $\mathcal{N}(0, 1)$ distribution, and an application of the continuity theorem completes the proof.
\end{proof}

\begin{corollary}
$Q_n = \frac{1}{n}S_n \to \mathcal{N} \left(\mu, \frac{\sigma^2}{n} \right)$, $S_n \to \mathcal{N}(\mu, n \sigma^2)$. The sampling error is proportional to $\frac{1}{\sqrt{n}}$.
\end{corollary}
There is a generalization. If $X_i$ is not i.i.d., we can still use the central limit theorem.

\end{document}