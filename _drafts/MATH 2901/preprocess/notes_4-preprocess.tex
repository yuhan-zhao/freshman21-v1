\documentclass[11pt]{article}
\usepackage[utf8]{inputenc}
\usepackage[shortlabels]{enumitem}
\usepackage{amsmath}
\usepackage{amssymb}
\usepackage{amsthm}
\usepackage{xcolor}
\usepackage{graphicx}
\usepackage{blkarray}

\usepackage{hyperref}
\hypersetup{
    colorlinks=true,
    linkcolor=blue,
    filecolor=magenta,      
    urlcolor=cyan,
}


% setting page format
\topmargin -.5in
\textheight 9in
\oddsidemargin -.25in
\evensidemargin -.25in
\textwidth 7in
\setlength{\parindent}{0 in}
\setlength{\parskip}{0.1 in}

% setting new environments
\newtheorem*{theorem}{Theorem}
\newtheorem*{lemma}{Lemma}
\newtheorem*{corollary}{Corollary}

\theoremstyle{definition}
\newtheorem*{definition}{Definition}
\newtheorem*{example}{Example}
\newtheorem*{problem}{Problem}
\newtheorem*{question}{Question}

\theoremstyle{remark}
\newtheorem*{remark}{Remark}

\newenvironment{solution}[1][Solution]{\textbf{#1:} \par}{\ $\blacksquare$}

\newenvironment{newnotion}[1]{\textbf{#1.}}

\newenvironment{caution}{$\bigstar\bigstar\bigstar$\textbf{Caution.}}

%\newenvironment{proof}[1][Proof]{\textbf{#1:} \par}{\ \rule{0.5em}{0.5em}}
%\newenvironment{problem}[1]{\textbf{#1:} }

% define new commands
\renewcommand{\hat}{\widehat}
\renewcommand{\tilde}{\widetilde}

\newcommand{\numpy}{{\tt numpy}}    % tt font for numpy
\newcommand{\dom}[1]{\mathbf dom #1}
\newcommand{\tr}{\mathbf{tr}}
\newcommand{\sgn}{\text{sgn}}
\newcommand{\spacevert}{\;\vert\;}
\newcommand{\ie}{i.e.}
\newcommand{\eg}{e.g.}

\newcommand{\N}{\mathbb{N}}
\newcommand{\E}{\mathbb{{E}}}
\newcommand{\Q}{\mathbb{Q}}
\newcommand{\R}{\mathbb{R}}
\newcommand{\C}{\mathbb{C}}
\newcommand{\Z}{\mathbb{Z}}
\newcommand{\mS}{\mathbb{S}}

% for probability
\newcommand{\Exp}{\mathbf{E}}
\newcommand{\Var}{\mathbf{Var}}
\newcommand{\Prob}{\mathbf{P}}
\newcommand{\card}{\mathbf{card}}
\newcommand{\cov}{\mathbf{cov}}
\newcommand{\corr}{\mathbf{corr}}
\newcommand{\1}{\mathbf{1}}


% for optimizations
\newcommand{\subto}{\text{s.t.}}

% for norms
\newcommand{\norm}[1]{\left\Vert #1 \right\Vert}
\newcommand{\abs}[1]{\left\vert #1 \right\vert}

\begin{document}
% ========== Edit your name here
\title{MATH 2901 Basic Probability Lecture Notes 4}
\author{Instructor: Richard Kleeman}
\date{}
\maketitle

%\medskip

% ========== Contents begin here ==============
## Expectation of discrete random variables
The intuition of expectation is the *average value* of an experiment. Suppose we do an experiment for $N$ repeated times. The probability of each possible outcome $x$ can be approximately defined by 
$$\begin{equation}
    \mathbf{P}(x) \approx f(x) = \frac{freq(x)}{N}.
\end{equation}$$
Then the average outcome is
$$\begin{equation}
    m \approx \frac{1}{N}\sum_{x} freq(x)x = \sum_x f(x)x
\end{equation}$$

<div class="theorem-block"><p><b>Definition.<b> 
The **mean value**, or **expectation**, or **expected value** of the random variable $X$ with mass function $f$ is defined to be 
$$\begin{equation}
    \mathbf{E}(X) = \sum_{x:f(x)>0} xf(x)
\end{equation}$$
whenever this sum is absolutely convergent.
</p></div>

<div class="remark-block"><p><b>Remark.<b> 
<ol>[(a)]
    <li> For notation convenience, we also write $\mathbf{E}(X) = \sum_x xf(x)$.
    <li> We require **absolute convergence** in order that $\mathbf{E}(X)$ be unchanged by reordering the $x_i$. 
</ol>
</p></div>

<div class="theorem-block"><p><b>Theorem.<b> [Riemann Rearrangement Theorem] \href{https://en.wikipedia.org/wiki/Riemann_series_theorem}{See this: Riemann series theorem}.
</p></div>

<div class="theorem-block"><p><b>Lemma.<b> 
If $X$ has mass function $f$ and $g:\mathbb{R}\to\mathbb{R}$, then
$$\begin{equation}
    \mathbf{E}(g(x)) = \sum_x g(x) f(x)
\end{equation}$$
whenever this sum is absolutely convergent.
</p></div>

<div class="remark-block"><p><b>Example.<b> 
If $X$ is a random variable with mass function f, and $g(x) = x^2$, then 
$$\begin{equation}
    \mathbf{E}(X^2) = \sum_x g(x)f(x) = x^2 f(x).
\end{equation}$$
</p></div>

<div class="theorem-block"><p><b>Definition.<b> 
If $k$ is a positive integer, the $k$th \text{moment} $m_k$ of $X$ is defined to be 
$$\begin{equation}
    m_k = \mathbf{E}(X^k).
\end{equation}$$
The $k$th **central moment** $\sigma_k$ is defined as
$$\begin{equation}
    \sigma_k = \mathbf{E}\left( (X-m_1)^k \right) = \mathbf{E}\left( (X-E(X))^k \right).
\end{equation}$$ 
</p></div>

The two moments of most use are $m_1 = \mathbf{E}(X)$ and $\sigma_2 = \mathbf{E}( (X - \mathbf{E}(X))^2$, called the **mean** (or **expectation)** and **variance** of $X$. These two quantities are measures of the mean and dispersion of $X$; that is, $m_1$ is the average value of $X$, and $\sigma_2$ measures the amount by which $X$ tends to deviate from this average. The mean $m_1$ is often denoted $\mu$, and the variance of $X$ is often denoted $\mathbf{Var}(X)$. The positive square root $\sigma = \mathbf{Var}(X)$ is called the **standard deviation**, and in this notation $\sigma_2 = \sigma^2$. 

The central moments $\{\sigma_i\}$ can be expressed in terms of the ordinary moments $\{m_i\}$. For example, $\sigma_1 = 0$, and 
$$\begin{equation}
    \begin{aligned} 
        \sigma_{2} &=\sum_{x}\left(x-m_{1}\right)^{2} f(x) \\ 
        &=\sum_{x} x^{2} f(x)-2 m_{1} \sum_{x} x f(x)+m_{1}^{2} \sum_{x} f(x) \\ 
        &=m_{2}-m_{1}^{2} ,
    \end{aligned}
\end{equation}$$
which may be written as 
$$$$$$$$\begin{equation}
    \label{eq:4-1}
    \tag{4-1}
    \mathbf{Var}(X) = \mathbf{E}\left( (X-E(X))^2 \right) = \mathbf{E}(X^2) - \mathbf{E}(X)^2.
\end{equation}$$$$$$$$

<div class="remark-block"><p><b>Example.<b> [Binomial variables]
Let $X$ be a random variable with binomial distribution. The p.m.f. is 
$$\begin{equation}
    f(k) = \binom{n}{k} p^k q^{n-k} \quad k = 0,\dots, n,
\end{equation}$$
where $q = 1-p$. The expectation of $X$ is
$$\begin{equation}
    \mathbf{E}(X) = \sum_{k=0}^n k f(k) = \sum_{k=0}^n k\binom{n}{k} p^k q^{n-k}.
\end{equation}$$
We use the following algebraic identity to compute $\mathbf{E}(X)$.
$$$$$$$$\begin{equation}
    \label{eq:4.2}
    \tag{4-2}
    \sum_{k=0}^n \binom{n}{k} x^k = (1+x)^n, 
\end{equation}$$$$$$$$
Differentiate it and multiply by $x$, we obtain 
$$$$$$$$\begin{equation}
    \label{eq:4.3}
    \tag{4-3}
    \sum_{k=0}^n k \binom{n}{k} x^k = nx(1+x)^{n-1}. 
\end{equation}$$$$$$$$
We substitute $x = p / q$ to obtain $\mathbf{E}(X) = np$. A similar argument shows that the variance of $X$ is given by $\mathbf{Var}(X) = npq$. 
</p></div>

We can think of the process of calculating expectations as a **linear operator** on the space of random variables. 

<div class="theorem-block"><p><b>Theorem.<b> 
The expectation operator $\mathbf{E}$ has the following properties: 
<ol>[(a)]
    <li> if $X\geq 0$, then $\mathbf{E}(X) \geq 0$,
    <li> if $a, b \in \mathbb{R}$, then $\mathbf{E}(aX+bY) = a\mathbf{E}(X) + b\mathbf{E}(Y)$,
    <li> the random variable 1, taking the value 1 always, has expectation $\mathbf{E}(1) = 1$. 
</ol>
</p></div>

\begin{proof}
We only prove the second property, which is also called the linear property.

We must use the joint p.m.f. of $X$ and $Y$ to compute the expectation. 
$$\begin{equation}
    \begin{split}
        \mathbf{E}(aX+bY) &= \sum_{i, j} (ax_i + by_j) f(x_i, y_j) \\
        &= a \sum_{i,j} x_i f(x_i, y_j) + b\sum_{i,j} y_j f(x_i, y_j) \\
        &= a\sum_{i} x_i f_X(x_i) + b\sum_{j}y_j f_Y(y_j) \\ 
        &= a\mathbf{E}(X) + b\mathbf{E}(Y),
    \end{split}
\end{equation}$$
where $f_X(x)$ and $f_Y(y)$ are marginal p.m.f. of $X$ and $Y$ respectively.
\end{proof}

\begin{caution}
It is **NOT** in general true that $\mathbf{E}(XY)$ is the same as $\mathbf{E}(X)\mathbf{E}(Y)$. 
\end{caution}

<div class="theorem-block"><p><b>Lemma.<b> 
If $X$ and $Y$ are independent, then $\mathbf{E}(XY) = \mathbf{E}(X)\mathbf{E}(Y)$. 
</p></div>
\begin{proof}
If $X, Y$ are independent, $f(x,y) = f_X(x) f_Y(y)$. Then
$$\begin{equation}
    \mathbf{E}(XY) = \sum_{ij} x_i y_j f(x,y) = \sum_{i} \left( x_i f_X(x_i) \right) \sum_{j} \left( y_j f_Y(y_j) \right) = \mathbf{E}(X) \mathbf{E}(Y).
\end{equation}$$
\end{proof}

<div class="theorem-block"><p><b>Definition.<b> 
$X$ and $Y$ are called **uncorrelated** if $\mathbf{E}(XY) = \mathbf{E}(X)\mathbf{E}(Y)$.
</p></div>

\begin{caution}
Independent variables are uncorrelated. But the converse is **NOT** true.
\end{caution}

<div class="theorem-block"><p><b>Theorem.<b> 
For random variables $X$ and $Y$, 
<ol>[(a)]
    <li> $\mathbf{Var}(aX) = a^2 \mathbf{Var}(X)$ for $a \in \mathbb{R}$,
    <li> $\mathbf{Var}(X+Y) = \mathbf{Var}(X) + \mathbf{Var}(Y)$ is $X$ and $Y$ are uncorrelated.
</ol>
</p></div>

<div class="remark-block"><p><b>Remark.<b> 
The above theorem shows that the variance operator $\mathbf{Var}$ is **NOT** a linear operator, even when it 
is applied only to uncorrelated variables. 
</p></div>

%==============================================
Sometimes the sum $S = \sum xf(x)$ does not converge absolutely, which means the mean of the distribution does not exist. Here is an example. 
<div class="remark-block"><p><b>Example.<b> 
**A distribution without a mean.** Let $X$ have mass function 
$$\begin{equation}
    f(k) = Ak^{-1} \quad k = \pm 1, \pm 2, \dots,
\end{equation}$$
where $A$ is chosen so that $\sum_k f(k) = 1$. The sum $\sum_k kf(k) = A\sum_{k\neq 0} k^{-1}$ doesn't converge absolutely, because both the positive and the negative parts diverge. 
</p></div>
This example is suitable to point out that we can base probability theory upon the expectation operator $\mathbf{E}$ rather than upon the probability measure $\mathbf{P}$. Roughly speaking, the way we proceed is to postulate axioms, such as (a)-(c) of the above Theorem, for a so-called ``expectation operator" $\mathbf{E}$ acting on a space of ``random variables". The probability of an event can then be recaptured by defining $\mathbf{P}(A) = \mathbf{E}(I_A)$.  

Recall the indicator function of a set $A$ is defined as
$$\begin{equation}
    I_A(\omega) = \begin{cases} 1 & \omega \in A, \\
    0 & \omega \not\in A. \end{cases}
\end{equation}$$
In addition, we have $\mathbf{E}(I_A) = \mathbf{P}(A)$.

%==============================================

## Dependence of discrete random variables
<div class="theorem-block"><p><b>Definition.<b> 
The **joint distribution function** $F:\mathbb{R}^2 \to [0,1]$ of $X$ and $Y$, where $X$ and $Y$ are discrete variables, is given by 
$$\begin{equation}
    F(x, y) = \mathbf{P}(X\leq x \text{ and } Y \leq y). 
\end{equation}$$
Their **joint mass function** $f:\mathbb{R}^2 \to [0,1]$ is given by 
$$\begin{equation}
    f(x,y) = \mathbf{P}(X = x \text{ and } Y = y). 
\end{equation}$$
</p></div>

We write $F_{X,Y}$ and $f_{X,Y}$ when we need to stress the role of $X$ and $Y$. We may think of the joint mass function in the following way. If $A_x = \{X = x\}$ and $B_y = \{Y = y\}$, then 
$$\begin{equation}
    f(x,y) = \mathbf{P}(A_x \cap B_y).
\end{equation}$$

<div class="theorem-block"><p><b>Lemma.<b> 
The discrete random variables $X$ and $Y$ are **independent** if and only if 
$$$$$$$$\begin{equation}
    \label{eq:4.4}
    \tag{4-4}
    f_{X,Y}(x,y) = f_X(x)f_Y(y) \quad \forall x,y \in \mathbb{R}.
\end{equation}$$$$$$$$
More generally, $X$ and $Y$ are independent if and only if $f_{X,Y}(x, y)$ can be **factorized as the product** $g(x)h (y)$ of a function of $x$ alone and a function of $y$ alone. 
</p></div>
<div class="remark-block"><p><b>Remark.<b> 
We stress that the factorization \eqref{eq:4.4} must hold for all $x$ and $y$ in order that $X$ and $Y$ be independent. 
</p></div>

<div class="theorem-block"><p><b>Lemma.<b> 
$$\begin{equation}
    \mathbf{E}(g(X, Y))=\sum_{x, y} g(x, y) f_{X, Y}(x, y).
\end{equation}$$
</p></div>

<div class="theorem-block"><p><b>Definition.<b> 
The **covariance** of $X$ and $Y$ is
$$\begin{equation}
    \mathbf{cov}(X,Y) = \mathbf{E}\left( (X-\mathbf{E}(X))(Y-\mathbf{E}(Y)) \right).
\end{equation}$$
The **correlation (coefficient)** of $X$ and $Y$ is 
$$\begin{equation}
\mathbf{corr}(X, Y) = \rho(X,Y) = \frac{\mathbf{cov}(X,Y)}{\sqrt{\mathbf{Var}(X)\mathbf{Var}(Y)}}
\end{equation}$$
as long as the variances are non-zero. 
</p></div>

<div class="remark-block"><p><b>Remark.<b> 
Notice the following two equations.
<ol>
    <li> $\mathbf{cov}(X,X) = \mathbf{Var}(X)$,
    <li> $\mathbf{cov}(X,Y) = \mathbf{E}(XY) - \mathbf{E}(X)\mathbf{E}(Y)$.
</ol>
</p></div>

Covariance itself is not a satisfactory measure of dependence because the scale of values which $\mathbf{cov}(X, Y)$ may take contains no points which are clearly interpretable in terms of the relationship between $X$ and $Y$.

<div class="theorem-block"><p><b>Theorem.<b> [Cauchy-Schwarz inequality] For random variables $X$ and $Y$, 
$$\begin{equation}
    \mathbf{E}(XY)^2 \leq \mathbf{E}(X^2) \mathbf{E}(Y^2)
\end{equation}$$ 
with equality if and only if $\mathbf{P}(aX = bY) = 1$ for some real $a$ and $b$, at least one of which is non-zero. 
</p></div>

\begin{proof}
For $a, b \in \mathbb{R}$, let $Z = aX - bY$. Then 
$$\begin{equation}
    0 \leq \mathbf{E}(Z^2) = a^2 \mathbf{E}(X^2) - 2ab\mathbf{E}(XY) + b^2\mathbf{E}(Y^2).
\end{equation}$$
Thus the right-hand side is a quadratic in the variable $a$ with at most one real root. Its discriminant must be non-positive. That is to say, if $b \neq 0$, 
$$\begin{equation}
    \mathbf{E}(XY)^2 - \mathbf{E}(X^2) \mathbf{E}(Y^2) \leq 0. 
\end{equation}$$
The discriminant is zero if and only if the quadratic has a real root. This occurs if and only if 
$$\begin{equation}
    \mathbf{E}\left( (aX-bY)^2 \right) = 0
\end{equation}$$
for some $a$ and $b$.
\end{proof}

We define $X' = X-\mathbf{E}(X), Y' = Y - \mathbf{E}(Y)$. Since all $X, Y$ satisfy the Cauchy-Schwarz inequality, so do $X'$ and $Y'$. Therefore, 
$$\begin{equation}
    \mathbf{E}(X'Y')^2 \leq \mathbf{E}(X'^2) \mathbf{E}(Y'^2) \quad \Leftrightarrow \quad \mathbf{cov}(X, Y)^2 \leq \mathbf{Var}(X)\mathbf{Var}(Y).
\end{equation}$$
Therefore, 
$$\begin{equation}
    \rho(X,Y)^2 \leq 1 \quad \mathbb{R}ightarrow \quad \rho(X,Y) \in [-1, 1].
\end{equation}$$
which gives the following lemma.

<div class="theorem-block"><p><b>Lemma.<b> 
The correlation coefficient $\rho$ satisfies $\left\vert \rho (X, Y)  \right\vert \leq 1$ with equality if and only if $\mathbf{P}(aX + bY = c) = 1$ for some $a, b, c \in \mathbb{R}$. 
</p></div>


## Expectation of continuous random variables
### Idea of translating expectation from discrete to continuous
Suppose we have a continuous random variable $X$ with $f$ being the probability density function. We split $X$ into small intervals $\Delta x$. Then $p_i = f(x_i)\Delta x$. $\frac{p_i}{\Delta x}$ is an approximation of probability density function. Therefore, 
$$\begin{equation}
    \mathbf{E}(X) \approx \sum_{i} x_i p_i = \sum_{i} x_i f(x_i) \Delta x,
\end{equation}$$
which is the Remann sum. We take the limit and get
$$\begin{equation}
    \mathbf{E}(x) = \int_{-\infty}^\infty x f(x)dx.
\end{equation}$$

### Expectation
<div class="theorem-block"><p><b>Definition.<b> 
The **expectation** of a continuous random variable $X$ with density function $f$ is given by 
$$\begin{equation}
    \mathbf{E}(X) = \int_{-\infty}^\infty xf(x) dx
\end{equation}$$
whenever this integral exists.
</p></div>

<div class="theorem-block"><p><b>Theorem.<b> 
If $X$ and $g(X)$ are continuous random variables, then $$\begin{equation}
    \mathbf{E}\left( g(X) \right) = \int_{-\infty}^\infty g(x)f(x) dx.
\end{equation}$$
</p></div>

<div class="theorem-block"><p><b>Definition.<b> 
The $k$th **moment** of a continuous variable $X$ is defined as
$$\begin{equation}
    \mathbf{E}(X^k) = \int_{-\infty}^\infty x^k f(x) dx
\end{equation}$$
whenever the integral converges.
</p></div>

<div class="remark-block"><p><b>Example.<b> [Cauchy distribution] The random variable $X$ has the Cauchy distribution t if it has density 
function 
$$\begin{equation}
    f(x) = \frac{1}{\pi (1+x^2)}, \quad x \in \mathbb{R}.
\end{equation}$$
This distribution is notable for having **no moments**.
</p></div>

## Dependence of continuous random variables
<div class="theorem-block"><p><b>Definition.<b> 
The **joint distribution function** of $X$ and $Y$ is the function $F: \mathbb{R}^2 \to [0, 1]$ given by 
$$\begin{equation}
    F(x,y) = \mathbf{P}(X\leq x, Y \leq y).
\end{equation}$$
</p></div>

<div class="theorem-block"><p><b>Definition.<b> 
The random variables $X$ and $Y$ are **(jointly) continuous** with **joint (probability) density function** $f : \mathbb{R}^2 \to [0, \infty)$ if
$$\begin{equation}
    F(x, y)=\int_{v=-\infty}^{y} \int_{u=-\infty}^{x} f(u, v) d u d v \quad \text{for each } x, y\in\mathbb{R}. 
\end{equation}$$
If $F$ is sufficiently differentiable at the point $(x , y)$, then we usually specify 
$$\begin{equation}
    f(x, y)=\frac{\partial^{2}}{\partial x \partial y} F(x, y).
\end{equation}$$
</p></div>

\begin{newnotion}{Probabilities}
$$\begin{equation}
    \begin{aligned} 
        \mathbf{P}(a \leq X \leq b, c \leq Y \leq d) &=F(b, d)-F(a, d)-F(b, c)+F(a, c) \\ 
        &=\int_{y=c}^{d} \int_{x=a}^{b} f(x, y) d x d y. \end{aligned}
\end{equation}$$
If $B$ is a sufficiently nice subset of $\mathbb{R}^2$, then
$$\begin{equation}
    \mathbf{P} \left( (X, Y) \in B \right)=\iint_{B} f(x, y) d x d y.
\end{equation}$$
\end{newnotion}
\begin{newnotion}{Marginal distributions}
The marginal distribution functions of $X$ and $Y$ are
$$\begin{equation}
    F_{X}(x)=\mathbf{P}(X \leq x)=F(x, \infty), \quad F_{Y}(y)=\mathbf{P}(Y \leq y)=F(\infty, y). 
\end{equation}$$
$$\begin{equation}
    F_{X}(x)=\int_{-\infty}^{x}\left(\int_{-\infty}^{\infty} f(u, y) d y\right) d u.
\end{equation}$$
Marginal density function of $X$ and $Y$:
$$\begin{equation}
    f_{X}(x)=\int_{-\infty}^{\infty} f(x, y) d y, \quad f_{Y}(y)=\int_{-\infty}^{\infty} f(x, y) d x.
\end{equation}$$
\end{newnotion}

\begin{newnotion}{Expectation}
If $g: \mathbb{R}^2 \to \mathbb{R}$ is a sufficiently nice function, then 
$$\begin{equation}
    \mathbf{E}(g(X, Y))=\int_{-\infty}^{\infty} \int_{-\infty}^{\infty} g(x, y) f(x, y) d x d y;
\end{equation}$$
in particular, setting $g(x, y) = ax + by$, 
$$\begin{equation}
    \mathbf{E}(aX+bY) = a\mathbf{E}(X) + b\mathbf{E}(Y).
\end{equation}$$
\end{newnotion}

\begin{newnotion}{Independence}
The random variables $X$ and $Y$ are independent if and only if 
$$\begin{equation}
    F(x,y) = F_X(x) F_Y(y) \quad \forall x, y \in \mathbb{R},
\end{equation}$$
which, for **continuous random variables**, is equivalent to requiring that 
$$\begin{equation}
    f(x,y) = f_X(x) f_Y(y).
\end{equation}$$
\end{newnotion}
 
<div class="theorem-block"><p><b>Theorem.<b> [Cauchy-Schwarz inequality] For any pair $X, Y$ of jointly continuous variables, we have that 
$$\begin{equation}
    \mathbf{E}(XY)^2 \leq \mathbf{E}(X^2) \mathbf{E}(Y^2), 
\end{equation}$$
with equality if and only if $\mathbf{P}(aX = bY) = 1$ for some real $a$ and $b$, at least one of which is non-zero. 
</p></div>


\end{document}